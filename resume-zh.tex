%%
%% Copyright (c) 2018-2019 Weitian LI <wt@liwt.net>
%% CC BY 4.0 License
%%
%% Created: 2018-04-11
%%

% Chinese version
\documentclass[zh]{resume}
\usepackage{./fontawesome5/fontawesome5}

% File information shown at the footer of the last page
\fileinfo{%
  % \faCopyright{} 2018--2019 Weitian LI,
  % \creativecommons{by}{4.0},
  % \githublink{liweitianux}{resume},
  % \faCalendarAlt{} \today
}

\name{伟业}{林}

%\taglineicon{\faBinoculars}
%\tagline{???}
\keywords{BSD, Linux, Programming, Python, C, Shell, DevOps, SysAdmin}

\socialinfo{
  \mobile{15625076252}
  \email{724063132@qq.com}
  \github{claviering}
  % \university{华南师范大学}
  % \degree{网络工程 \textbullet 本科}
  % \home{广东 \textbullet 广州}
  \birthday{1997-02}
}

\begin{document}
\makeheader

%======================================================================
% Summary & Objectives
%======================================================================
{\onehalfspacing\hspace{2em}%
计算机专业,有网络、安全学与密码学基础,擅长前端工程开发,热衷前端技术,有 Linux 和 Mac 使用经验,实践应用过前端框架 Vue、React 和 Angular。有 Node 开发经验,有小程序开发经验,有了解 webGL 和 threejs。热爱学习,关注最新前端技术
\par}

%======================================================================
% \sectionTitle{技能和语言}{\faWrench}
%======================================================================
% \begin{competences}
%   \comptence{编程}{%
%     C, JavaScript, Python, TypeScript
%   }
%   \comptence{工具}{%
%     Git, Make, VSCode, Jenkins
%   }
%   \comptence{\icon{\faLanguage} 语言}{
%     \textbf{英语} --- 大学四级
%   }
% \end{competences}

%======================================================================
\sectionTitle{教育背景}{\faGraduationCap}
%======================================================================
\begin{educations}
  \education%
    {2015.09}%
    [2019.07]%
    {华南师范大学}%
    {计算机学院}%
    {网络工程}%
    {本科}
\end{educations}

%======================================================================
\sectionTitle{前端开发技能}{\faCogs}
%======================================================================
\begin{itemize}
  \item 熟悉使用 react vue 技术栈开发应用,并配合 Webpack 打包 web 端和移动端项目
  \item 使用 NodeJS 搭建服务器,使用过 Express, Koa 和 Egg,操作 Redis 和 MongoDB 数据库
  \item 组件化思想,代码复用,良好的代码风格,能配合设计师完成 UI 设计
\end{itemize}

%======================================================================
\sectionTitle{个人项目}{\faCode}
%======================================================================
\begin{itemize}
  % \item \link{https://github.com/claviering/koa-home}{\texttt{koa-home}}:
  %   (NodeJS, Koa)
  %   Koa 后端 Demo。基本的 session redis log mongo socket 中间件
  % \item \link{https://github.com/claviering/express-home}{\texttt{express-home}}:
  %   (NodeJS, Express)
  %   express 后端 Demo
  % \item \link{https://github.com/claviering/nodejs_sso}{\texttt{nodejs\_sso}}:
  %   (Nodejs)
  %   Nodejs 实现 sso
  % \item \link{https://github.com/claviering/webpack4-config}{\texttt{webpack4-config}}:
  %   (Webpack 4)
  %   Wepack 4 加 balel 7 配置 Vue、React 和 TS
  % \item \link{https://github.com/claviering/wepy-demo}{\texttt{wepy-demo}}:
  %   (wepy)
  %   全栈小程序 demo。实现基本路由,状态管理,Websocket
  \item \link{https://github.com/claviering/myney}{\texttt{myney}}:
  微信小程序-记账小程序,使用云函数开发。微信搜索 myney 体验
  \item \link{https://codepen.io/claviering}{\texttt{codepen}}:
  codepen 上的一些 demo

\end{itemize}

%======================================================================
\sectionTitle{工作经历}{\faBriefcase}
%======================================================================
\begin{experiences}
  \experience%
    [2018.10]%
    {至今}%
    {前端开发工程师 @ 赫基集团}%
    [\begin{itemize}
      \item CRM 项目。\\ 技术栈是 react 和 Koa \\ 客户关系管理,导购和会员的管理,事件关怀,活动推送等,主要负责有:\\ 1. 基于antd design 组件封装业务组件。\\ 2. nodejs 封装多个业务接口,抽象出共用接口。
      % \item MA 项目。主要有: \\ 1. 功能模块的权限控制 \\ 2. 会员功能。分为注册会员,升级会员,消费会员模块
      \item SSC 项目。\\ 技术栈是 anglar 1.3。\\ 门店管理端,管理员用于配置门店。主要负责有促销活动模块
      % \item 收银端。文字消息3种跑马灯效果实现。
      \item 微分销小程序。\\ 使用原生开发 \\ 购物小程序,主要负责有活动页面开发,客户引流页面开发。
      \item ITOPS 运营平台。\\ 技术栈是 vue 和 eggjs。\\ 统一管理多个应用系统。主要负责有消息中心开发,使用 websocket 推送消息到其他应用。消息保存到 mongodb。设计消息实体,开发 Eggjs 插件实现应用重启初始化定时任务。各应用加入不同的 room,实现多应用对接
    \end{itemize}]

  % \separator{0.5ex}
  % \experience%
  %   [2018.10]%
  %   {2019.04}%
  %   {前端开发工程师(实习) @ 赫基集团}%
  %   [\begin{itemize}
  %     \item CRM 项目开发,Web 端,使用 React,NodeJS 做中台。实现优惠券、会员相关等功能。
  %     \item MA 项目开发,移动端收银项目。使用 Vue 框架。商品功能,会员功能等。
  %   \end{itemize}]
\end{experiences}

\end{document}
